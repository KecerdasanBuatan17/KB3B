\section{1174040 - Hagan Rowlenstino A. S}
    \subsection{Teori}
        \begin{enumerate}
            \item Jelaskan dengan ilustrasi gambar sendiri apa itu generator dengan perumpamaan anda sebagai mahasiswa sebagai generatornya.
            \par Generator adalah sebuah jaringan yang merubah inputan vektor menjadi gambar, seperti ada inputan vektor secara acak yaitu angka sembarang, maka angka tersebut akan diubah menjadi gambar yang sembarang pula. Sebagai ilustrasi apabila mahasiswa sebagai generator, mmisalkan inputan vektor tersebut adalah bahan bahan membuat kue, maka hasil dari generator tersebut juga akan acak, bisa kue bolu, cupcake, ataupun kue lainnya.
            \begin{figure}[H]
                \includegraphics[width=4cm]{figures/1174040/chapter8/teori1.png}
                \centering
                  \caption{Teori No 1}
            \end{figure}

            \item Jelaskan dengan ilustrasi gambar sendiri apa itu diskriminator dengan perumpamaan dosen anda sebagai diskriminatornya
            \par Diskriminator adalah sebuah jaringan yang mengeluarkan klasifikasi untuk menyatakan input gambar adalah asli dari dataset atau buatan dari generator. Lebih mudahnya diskriminator digunakan agar hasil gambar dari generator sesuai dengan yang diinginkan. sebagai ilustrasi nya dosen menyuruh mahasiswanya membuat kue bolu, maka hasil yang akan dibuat adalah kue bolu.
            \begin{figure}[H]
                \includegraphics[width=4cm]{figures/1174040/chapter8/teori2.png}
                \centering
                  \caption{Teori No 2}
            \end{figure}

            \item Jelaskan dengan ilustrasi gambar sendiri bagaimana arsitektur generator dibua
            \par Generator menggunkan input array random yang bernama seed, dimana akan diubah menjadi sebuah gambar dengan menggunakan Convolutional Neural Network yang dapat dilihat pada gambar.
            \begin{figure}[H]
                \includegraphics[width=4cm]{figures/1174040/chapter8/teori3.png}
                \centering
                  \caption{Teori No 3}
            \end{figure}

            \item Jelaskan dengan ilustrasi gambar sendiri bagaimana arsitektur diskriminator dibuat
            \par Diskriminator adalah CNN yang menerima inputan image lalu menghasilkan angka biner yang dapat menyatakan apakah gambar tersebut asli atau sesuai dengan dataset asli
            \begin{figure}[H]
                \includegraphics[width=4cm]{figures/1174040/chapter8/teori4.png}
                \centering
                  \caption{Teori No 4}
            \end{figure}

            \item Jelaskan dengan ilustrasi gambar apa itu latent space
            \par latent space adalah representasi dari data yang di kompress, untuk contohnya misalnya ada 3 gambar yaitu 2 kursi yang berebeda dan 1 meja, maka hal hal yang mirip dai kursi tersebut(ciri-ciri utamanya), seeprti itulah latent space
            \begin{figure}[H]
                \includegraphics[width=4cm]{figures/1174040/chapter8/teori5.png}
                \centering
                  \caption{Teori No 5}
            \end{figure}

            \item apa itu adversarial play
            \par adversarial play adalah dimana para jaringan di latih, dimana jaringan satu dan lainnya saling berkompetisi. dapat disimpulkan dimana jaringan generator dan jaringan discriminator saling bertemu berulang ulang kali
            
            \item Jelaskan dengan ilustrasi gambar apa itu Nash equilibrium
            \par Nash equilibrium adalah konsep dalam teori permainan di mana hasil optimal dari permainan adalah di mana tidak ada insentif untuk menyimpang dari strategi awal mereka. Lebih khusus lagi, keseimbangan Nash adalah konsep teori permainan di mana hasil optimal dari permainan adalah di mana tidak ada pemain yang memiliki insentif untuk menyimpang dari strategi yang dipilihnya setelah mempertimbangkan pilihan lawan.
            \begin{figure}[H]
                \includegraphics[width=4cm]{figures/1174040/chapter8/teori7.png}
                \centering
                  \caption{Teori No 7}
            \end{figure}

            \item Sebutkan dan jelaskan contoh-contoh implementasi dari GAN
            \par Pada bidang mode, seni, dan iklan, GAN dapar digunakan untuk membuat foto-foto model fashion imajinier tanpa perlu menyewa model. Pada bidang sains GAN dapat meningkatkan citra astronomi dan mensimulasikan pelensaan gravitasi untuk penelitian materi gelap.

            \item Berikan contoh dengan penjelasan kode program beserta gambar arsitektur untuk membuat generator(neural network) dengan sebuah input layer, tiga hidden layer(dense layer), dan satu output layer(reshape layer)
\begin{verbatim}
gen=Sequential() #Inisiasi dari sequensial
gen.add(Dense(units=200,input_dim=np.shape(train_input)[1])) #Menambah dense layer dengan batch size 200 dan input dim dari input
gen.add(Dense(units=400))#Menambah dense layer dengan batch size 400 dan input dim 100
gen.add(Dense(units=784, activation='tanh')) #Menambah dense layer dengan batch size 784 dan aktivasi metode tanh
gen.compile(loss='binary_crossentropy', optimizer=adam_optimizer()) #Menkompilasi hasil penambahan setiap dense
 gen.summary() #Memproses data yang sudah disetting dan menampilkannya
\end{verbatim}    
            \par Pada contoh tersebut, data akan diambil dari hasil proses sebelumnya yaitu proses ekstrasi data gambar. Dari contoh ini, terdapat 3 layer dense, 1 input layer dan 1 output layer. Dari input layer nanti akan dimasukkan terlebih dahulu ke dense layer pertama lalu diproses oleh 2 layer selanjutnya dan terakhir akan ditampilkan oleh layer output.
            \begin{figure}[H]
                \includegraphics[width=4cm]{figures/1174040/chapter8/teori9.png}
                \centering
                  \caption{Teori No 9}
            \end{figure}

            \item Berikan contoh dengan ilustrasi dari arsitektur dikriminator dengan sebuath input layer, 3 buah hidden layer, dan satu output layer.
\begin{verbatim}
diskrim=Sequential()#Inisiasi dari sequensial
diskrim.add(Dense(units=784,input_dim=np.shape(train_input)[1]))#Menambah dense layer dan input dim dari layer
diskrim.add(Dense(units=400)) #Mensetting Dense
diskrim.add(Dense(units=200, activation='sigmoid')) #Mensetting dense dan melakukan aktivasi dengan metode sigmoid
diskrim.compile(loss='binary_crossentropy', optimizer=adam_optimizer())#Menkompilasi hasil penambahan setiap dense
diskrim.summary()#Memproses data yang sudah disetting dan menampilkannya
\end{verbatim}
            \par Pada contoh tersebut, data akan diambil dari hasil proses sebelumnya yaitu proses ekstrasi data gambar. Dari contoh ini, terdapat 3 layer dense, 1 input layer dan 1 output layer. Pada proses ini, seluruh data akan dibandingkan dengan data sebelumnya yaitu dari generator dan dari data aslinya yang sudah dijadikan data vector. 
            \begin{figure}[H]
                \includegraphics[width=4cm]{figures/1174040/chapter8/teori10.png}
                \centering
                  \caption{Teori No 10}
            \end{figure}
            
            \item Jelaskan bagaimana kaitan output dan input antara generator dan diskriminator tersebut. Jelaskan kenapa inputan dan outputan seperti itu
            \par Pada kedua metode tersebut, akan disebutkan berapa akurasi dari setiap metode. Pada setiap metode tersebut (Discriminator dan generator) akan dilakukan pelatihan dan akan dibandingkan hasilnya. Generator akan menghasilkan data baru sesuai dengan hasil latihan dan dari data tersebut, discriminator akan membandingkan dengan data set apakah data tersebut "asli" atau tidak.
            
            \item Jelaskan apa perbedaan antara Kullback-Leibler divergence (KL divergence)/relative entropy, Jensen-Shannon(JS) divergence / information radius(iRaD) / total divergence to the average dalam mengukur kualitas dari model
            \par relative entropy adalah ukuran dari bagaimana satu distribusi probabilitas berbeda dari yang kedua, distribusi probabilitas referensi, Divergensi Jensen-Shannon adalah ukuran divergensi berprinsip yang selalu terbatas untuk variabel acak terbatas.

            \item Jelaskan apa itu fungsi objektif yang berfungsi untuk mengukur kesamaan antara gambar yang dibuat dengan yang asli.
            \par Fungsi objektif adalah fungsi yang digunakan sebagai penujuk berapa nilai kesamaan anatara gambar yang dibuat dengan yang asli

            \item Jelaskan apa itu scoring algoritma selain mean square error atau cross entropy seperti The Inception Score dan The Frechet Inception distance.
            \par Inception Score digunakan untuk mengukur seberapa realistis output dari GAN, dimana ada dua parameter, yaitu : gambarnya punya variasi dan setiap gambar jelas terlihat seperti sesuatu. Frechet Inception Distance adalah ukuran kesamaan antara dua dataset gambar. Itu terbukti berkorelasi baik dengan penilaian manusia terhadap kualitas visual dan paling sering digunakan untuk mengevaluasi kualitas sampel Generative Adversarial Networks. FID dihitung dengan menghitung jarak Fréchet antara dua Gaussians dipasang ke representasi fitur dari jaringan Inception.
            
            \item Jelaskan kelebihan dan kekurangan GAN
            \begin{itemize}
              \item Kelebihan 
              \begin{enumerate}
               
                \item GAN Menghasilkan data baru yang bisa hampir mirip dengan data asli. Karena hasil pelatihannya, GAN dapat menghasilkan data gambar, teks, audio, dan video yang dapat dibilang hampir mirip dengan yang aslinya. Berkat hal tersebut, GAN dapat digunakan dalam sistem marketing, e-commerce, games,iklan, dan industri lainnya
                \item GAN mempelajari representasi data secara internal sehingga beberapa masalah pada machine learning dapat diatasi dengan mudah
                \item Discriminator yang sudah dilatih dapat menjadi sebuah classifier atau pendeteksi jika data sudah sesuai. Karena Discriminator yang akan menjadi tidak efisien berkat seringnya dilatih
                \item GAN dapat dilatih menggunakan data yang belum dilabeled
              
            \end{enumerate}
              \item Kekurangan
              \begin{enumerate}  
                \item Data saat diproses oleh metode gan tidak konvergensi
                \item Jenis sampel yang dihasilkan oleh generator terbatas karena modenya terbatas
                \item Ketidak seimbangnya antara generator dan discriminator dapat menyebabkan overfitting atau terlalu dekat dengan hasil sampel
                \item Sangat sensitif dengan data yang sudah diinisiasi sebelumnya
              \end{enumerate}
            \end{itemize}

        \end{enumerate}
